\documentclass[11pt]{article}

% Language setting
% Replace `english' with e.g. `spanish' to change the document language
\usepackage[english]{babel}

% Set page size and margins
% Replace `letterpaper' with `a4paper' for UK/EU standard size
\usepackage[letterpaper,top=2cm,bottom=2cm,left=3cm,right=3cm,marginparwidth=1.75cm]{geometry}

% Useful packages
\usepackage{amsmath}
\usepackage{amssymb}
\usepackage{graphicx}
\usepackage[colorlinks=true, allcolors=blue]{hyperref}
% \usepackage{indentfirst}

\title{MATH 2600--Problem Set 3}
\author{Daniel Park}

\begin{document}
\maketitle

\section*{1. All Possible Products}
A is of shape \(3 \times 2\), B is of shape \(2 \times 4\), C is of shape \(4 \times 3\), and D is of shape \(3 \times 3\). Thus, the possible products are: \(AB\), \(BC\), \(CD\), \(DA\), \(CA\).
\[
    AB = \begin{pmatrix}
        2(1) + 5(9) & 2(7) + 5(2) & 2(2) + 5(7) & 2(9) + 5(1) \\
        1(1) + 4(9) & 1(7) + 4(2) & 1(2) + 4(7) & 1(9) + 4(1) \\
        2(1) + 1(9) & 2(7) + 1(2) & 2(2) + 1(7) & 2(9) + 1(1)
    \end{pmatrix}
\]
\[
    = \begin{pmatrix}
        47 & 24 & 39 & 23 \\
        37 & 15 & 30 & 13 \\
        11 & 16 & 11 & 19
    \end{pmatrix}
\]
\[
    BC = \begin{pmatrix}
        [1(1) + 7(2) + 2(1) + 9(3)] & [1(0) + 7(1) + 2(1) + 9(2)] & [1(4) + 7(3) + 2(5) + 9(1)] \\
        [9(1) + 2(2) + 7(1) + 1(3)] & [9(0) + 2(1) + 7(1) + 1(2)] & [9(4) + 2(3) + 7(5) + 1(1)] 
    \end{pmatrix}
\]
\[
    = \begin{pmatrix}
        44 & 27 & 44 \\
        23 & 11 & 78 
    \end{pmatrix}
\]
\[
    CD = \begin{pmatrix}
        [1(1) + 0(2) + 4(1)] & [1(0) + 0(1) + 4(3)] & [1(7) + 0(2) + 4(0)] \\
        [2(1) + 1(2) + 3(1)] & [2(0) + 1(1) + 3(3)] & [2(7) + 1(2) + 3(0)] \\
        [1(1) + 1(2) + 5(1)] & [1(0) + 1(1) + 5(3)] & [1(7) + 1(2) + 5(0)] \\
        [3(1) + 2(2) + 1(1)] & [3(0) + 2(1) + 1(3)] & [3(7) + 2(2) + 1(0)] 
    \end{pmatrix}
\]
\[
    = \begin{pmatrix}
        5 & 12 & 7 \\
        7 & 10 & 16 \\
        8 & 16 & 9 \\
        8 & 5 & 25
    \end{pmatrix}
\]
\[
    DA = \begin{pmatrix}
        [1(2) + 0(1) + 7(2)] & [1(5) + 0(4) + 7(1)] \\
        [2(2) + 1(1) + 2(2)] & [2(5) + 1(4) + 2(1)] \\
        [1(2) + 3(1) + 0(2)] & [1(5) + 3(4) + 0(1)]
    \end{pmatrix}
\]
\[
    = \begin{pmatrix}
        16 & 12 \\
        9 & 16 \\
        5 & 17
    \end{pmatrix}
\]
\[
    CA = \begin{pmatrix}
        [1(2) + 0(1) + 4(2)] & [1(5) + 0(4) + 4(1)] \\
        [2(2) + 1(1) + 3(2)] & [2(5) + 1(4) + 3(1)] \\
        [1(2) + 1(1) + 5(2)] & [1(5) + 1(4) + 5(1)] \\
        [3(2) + 2(1) + 1(2)] & [3(5) + 2(4) + 1(1)]  
    \end{pmatrix}
\]
\[
    = \begin{pmatrix}
        10 & 9 \\
        11 & 17 \\
        13 & 14 \\
        10 & 24
    \end{pmatrix}
\]

\section*{2. \(AB\) vs. \(BA\)}

\subsection*{2b. A = \(
    \begin{pmatrix}
        1 & -1 & 5 \\
        3 & 0 & 4
    \end{pmatrix}
    \), B = \(
        \begin{pmatrix}
            2 & 1 \\
            3 & 6 \\
            1 & 5
        \end{pmatrix}\)}
A is of shape \(2 \times 3\) and B is of shape \(3 \times 2\). Both \(AB\) and \(BA\) are thus defined. \(AB\) is of shape \(2 \times 2\) and \(BA\) is of shape \(3 \times 3\), so they do not have the same number of rows and columns.

\subsection*{2d. A = \(
    \begin{pmatrix}
        3 & 1 & -4 \\
        -2 & 0 & 5 \\
        1 & -2 & 3
    \end{pmatrix}
    \), B = \(
        \begin{pmatrix}
            2 & 0 & 0 \\
            0 & 5 & 0 \\
            0 & 0 & -1
        \end{pmatrix}\)}
A is of shape \(3 \times 3\) and B is of shape \(3 \times 3\). Both \(AB\) and \(BA\) are thus defined. \(AB\) is of shape \(3 \times 3\) and \(BA\) is of shape \(3 \times 3\), so they \textit{do} have the same number of rows and columns.

\[
    AB = \begin{pmatrix}
        [3(2) + 1(0) + -4(0)] & [3(0) + 1(5) + -4(0)] & [3(0) + 1(0) + -4(-1)] \\
        [-2(2) + 0(0) + 5(0)] & [-2(0) + 0(5) + 5(0)] & [-2(0) + 0(0) + 5(-1)]\\
        [1(2) + -2(0) + 3(0)] & [1(0) + -2(5) + 3(0)] & [1(0) + -2(0) + 3(-1)]
    \end{pmatrix}
\]
\[
    = \begin{pmatrix}
        6 & 5 & 4 \\
        -4 & 0 & -5 \\
        2 & -10 & -3
    \end{pmatrix}
\]
Since both \(A\) and \(B\) are square matrices and \(AB \neq I_3\), \(B \neq A^{-1}\), and thus \(BA\) is not going to equal to \(AB\). To check\dots
\[
    BA = \begin{pmatrix}
        [2(3) + 0(-2) + 0(1)] & [2(1) + 0(0) + 0(-2)] & [2(-4) + 0(5) + 0(3)] \\
        [0(3) + 5(-2) + 0(1)] & [0(1) + 5(0) + 0(-2)] & [0(-4) + 5(5) + 0(3)] \\
        [0(3) + 0(-2) + -1(1)] & [0(1) + 0(0) + -1(-2)] & [0(-4) + 0(5) + -1(3)]
    \end{pmatrix}
\]
\[
    = \begin{pmatrix}
        6 & 2 & -8 \\
        -10 & 0 & 25 \\
        -1 & 2 & -3
    \end{pmatrix}
\]
Indeed, \(AB \neq BA\).\pagebreak

\section*{3. \(A =
    \begin{pmatrix}
        2 & -5 \\
        3 & 1
    \end{pmatrix}
    \)
    }

\subsection*{3a. Evaluate \(A^2\)}
\[
    A^2 = \begin{pmatrix}
            [2(2) + -5(3)] & [2(-5) + -5(1)] \\
            [3(2) + 1(3)] & [3(-5) + 1(1)]
    \end{pmatrix}
\]
\[
    = \begin{pmatrix}
        -11 & -15 \\
        9 & -14
    \end{pmatrix}
\]

\subsection*{3b. \(\alpha\), \(\beta\), \(\gamma\)}
Find \(\alpha\), \(\beta\), \(\gamma\) \(\in \mathbb{R}\), such that \(\alpha I + \beta A + \gamma A^2\) = \textbf{0} and \(\alpha\), \(\beta\), \(\gamma\) not all 0.
\[
    \alpha I_2 + \beta A + \gamma A^2 = \textbf{0}
\]
\[
    \begin{pmatrix}
        \alpha & 0 \\
        0 & \alpha
    \end{pmatrix} + 
    \begin{pmatrix}
        2\beta & -5\beta \\
        3\beta & \beta
    \end{pmatrix} +
    \begin{pmatrix}
        -11\gamma & 15\gamma \\
        9\gamma & -14\gamma
    \end{pmatrix} = 
    \begin{pmatrix}
        0 & 0 \\
        0 & 0
    \end{pmatrix}
\]
\[
    \begin{pmatrix}
        \alpha + 2\beta - 11\gamma & -5\beta + 15\gamma \\
        3\beta + 9\gamma & \alpha + \beta - 14\gamma
    \end{pmatrix} = 
    \begin{pmatrix}
        0 & 0 \\
        0 & 0
    \end{pmatrix}
\]
\[
    \alpha + 2\beta - 11\gamma = 0
\]
\[
    -5\beta + 15\gamma = 0
\]
\[
    3\beta + 9\gamma = 0
\]
\[
    \alpha + \beta - 14\gamma = 0
\]
From the 2nd and 3rd equations, solving for \(\gamma\)\dots
\[
    \beta = -3\gamma
\]
Plugging that into the 4th equation\dots
\[
    \alpha - 17\gamma = 0
\]
\[
    \alpha = 17\gamma
\]
\[
    \gamma = \gamma
\]
There are infinitely many solutions. If not all zero, one solution is \(\alpha = 17\), \(\beta = -3\), \(\gamma = 1\). \pagebreak

\section*{5. Proof}
If \(A\) and \(B\) are matrices s.t. \(I - AB\) is invertible, then the inverse of \(I - BA\) is given by the formula \((I - BA)^{-1} = I + B(I - AB)^{-1}A\).\hfill\break
Let's denote \(C\) as \((I - AB)^{-1}\). Then,
\[
    (I - BA)^{-1} = I + BCA
\]
\[
    (I - BA)(I - BA)^{-1} = (I - BA)(I + BCA)
\]
We will prove that \((I - BA)(I + BCA) = I\), since \((I - BA)(I - BA)^{-1} = I\).
\[
    (I - BA)(I + BCA) = I
\]
\[
    I^2 + IBCA - BAI + BABCA = I
\]
\[
    I + BCA - BA + BABCA = I
\]
\[
    I + B(C - I - ABC)A = I
\]
\[
    I + B(C - ABC - I)A = I
\]
\[
    I + B[(I - AB)C - I]A = I
\]
Since \(C = (I - AB)^{-1}\), \(C^{-1} = (I - AB)\), because \(I - AB\) is assumed to be invertible.
\[
    I + B(C^{-1}C - I)A = I
\]
\[
    I + B(I - I)A = I
\]
\[
    I + B\textbf{0}A = I
\]
\[
    I + \textbf{0} = I
\]
\[
    I = I
\]
Thus, the inverse of \(I - BA\) is given by \(I + B(I - AB)^{-1}A\). $\blacksquare$ \pagebreak

\section*{6. Finding inverses}
\subsection*{6a. \(A = 
    \begin{bmatrix}
        1 & 1 & 1 \\
        1 & -1 & 1 \\
        0 & 0 & 1
    \end{bmatrix}
    \)}
\[
    \left[
    \begin{array}{ccc|ccc}
        1 & 1 & 1 & 1 & 0 & 0 \\
        1 & -1 & 1 & 0 & 1 & 0 \\
        0 & 0 & 1 & 0 & 0 & 1
    \end{array}
    \right]
\]
\[
    R_2 + (-R_1) = 
    \left[
    \begin{array}{ccc|ccc}
        0 & -2 & 0 & -1 & 1 & 0
    \end{array}
    \right]
\]
\[
    \left[
    \begin{array}{ccc|ccc}
        1 & 1 & 1 & 1 & 0 & 0 \\
        0 & -2 & 0 & -1 & 1 & 0 \\
        0 & 0 & 1 & 0 & 0 & 1
    \end{array}
    \right]
\]
\[
    \frac{R_2}{-2} =
    \left[
    \begin{array}{ccc|ccc}
        0 & 1 & 0 & \frac{-1}{-2} & \frac{1}{-2} & 0
    \end{array}
    \right]
\]
\[
    \left[
    \begin{array}{ccc|ccc}
        1 & 1 & 1 & 1 & 0 & 0 \\
        0 & 1 & 0 & \frac{1}{2} & -\frac{1}{2} & 0 \\
        0 & 0 & 1 & 0 & 0 & 1
    \end{array}
    \right]
\]
\[
    R_1 + (-R_2) = 
    \left[
    \begin{array}{ccc|ccc}
        1 & 0 & 1 & \frac{1}{2} & \frac{1}{2} & 0
    \end{array}
    \right]
\]
\[
    \left[
    \begin{array}{ccc|ccc}
        1 & 0 & 1 & \frac{1}{2} & \frac{1}{2} & 0 \\
        0 & 1 & 0 & \frac{1}{2} & -\frac{1}{2} & 0 \\
        0 & 0 & 1 & 0 & 0 & 1
    \end{array}
    \right]
\]
\[
    R_1 + (-R_3) = 
    \left[
    \begin{array}{ccc|ccc}
        1 & 0 & 0 & \frac{1}{2} & \frac{1}{2} & -1
    \end{array}
    \right]
\]
\[
    \left[
    \begin{array}{ccc|ccc}
        1 & 0 & 0 & \frac{1}{2} & \frac{1}{2} & -1 \\
        0 & 1 & 0 & \frac{1}{2} & -\frac{1}{2} & 0 \\
        0 & 0 & 1 & 0 & 0 & 1
    \end{array}
    \right]
\]
\[A^{-1} = 
    \begin{bmatrix}
        \frac{1}{2} & \frac{1}{2} & -1 \\
        \frac{1}{2} & -\frac{1}{2} & 0 \\
        0 & 0 & 1
    \end{bmatrix}
\]\pagebreak

\subsection*{6b. B = \(
    \begin{bmatrix}
        1 & 2 & -2 \\
        1 & 5 & 3 \\
        2 & 6 & -1
    \end{bmatrix}
    \)}
\[
    \left[
    \begin{array}{ccc|ccc}
        1 & 2 & -2 & 1 & 0 & 0 \\
        1 & 5 & 3 & 0 & 1 & 0 \\
        2 & 6 & -1 & 0 & 0 & 1
    \end{array}
    \right]
\]
\[
    R_2 + (-R_1) = 
    \left[
    \begin{array}{ccc|ccc}
        0 & 3 & 5 & -1 & 1 & 0
    \end{array}
    \right]
\]
\[
    \left[
    \begin{array}{ccc|ccc}
        1 & 2 & -2 & 1 & 0 & 0 \\
        0 & 3 & 5 & -1 & 1 & 0 \\
        2 & 6 & -1 & 0 & 0 & 1
    \end{array}
    \right]
\]
\[
    R_3 + (-2R_1) = 
    \left[
    \begin{array}{ccc|ccc}
        0 & 2 & 3 & -2 & 0 & 1
    \end{array}
    \right]
\]
\[
    \left[
    \begin{array}{ccc|ccc}
        1 & 2 & -2 & 1 & 0 & 0 \\
        0 & 3 & 5 & -1 & 1 & 0 \\
        0 & 2 & 3 & -2 & 0 & 1
    \end{array}
    \right]
\]
\[
    \frac{R_2}{3} = 
    \left[
    \begin{array}{ccc|ccc}
        0 & 1 & \frac{5}{3} & \frac{-1}{3} & \frac{1}{3} & 0
    \end{array}
    \right]
\]
\[
    \left[
    \begin{array}{ccc|ccc}
        1 & 2 & -2 & 1 & 0 & 0 \\
        0 & 1 & \frac{5}{3} & -\frac{1}{3} & \frac{1}{3} & 0 \\
        0 & 2 & 3 & -2 & 0 & 1
    \end{array}
    \right]
\]
\[
    R_3 + (-2R_2) = 
    \left[
    \begin{array}{ccc|ccc}
        0 & 0 & \frac{-1}{3} & \frac{-4}{3} & \frac{-2}{3} & 1
    \end{array}
    \right]
\]
\[
    \left[
    \begin{array}{ccc|ccc}
        1 & 2 & -2 & 1 & 0 & 0 \\
        0 & 1 & \frac{5}{3} & -\frac{1}{3} & \frac{1}{3} & 0 \\
        0 & 0 & -\frac{1}{3} & -\frac{4}{3} & -\frac{2}{3} & 1
    \end{array}
    \right]
\]
\[
    R_1 + (-2R_2) = 
    \left[
    \begin{array}{ccc|ccc}
        1 & 0 & \frac{-16}{3} & \frac{5}{3} & \frac{-2}{3} & 0
    \end{array}
    \right]
\]
\[
    \left[
    \begin{array}{ccc|ccc}
        1 & 0 & -\frac{16}{3} & \frac{5}{3} & -\frac{2}{3} & 0 \\
        0 & 1 & \frac{5}{3} & -\frac{1}{3} & \frac{1}{3} & 0 \\
        0 & 0 & -\frac{1}{3} & -\frac{4}{3} & -\frac{2}{3} & 1
    \end{array}
    \right]
\]
\[
    -3R_3 = 
    \left[
    \begin{array}{ccc|ccc}
        0 & 0 & 1 & 4 & 2 & -3
    \end{array}
    \right]
\]
\[
    \left[
    \begin{array}{ccc|ccc}
        1 & 0 & -\frac{16}{3} & \frac{5}{3} & -\frac{2}{3} & 0 \\
        0 & 1 & \frac{5}{3} & -\frac{1}{3} & \frac{1}{3} & 0 \\
        0 & 0 & 1 & 4 & 2 & -3
    \end{array}
    \right]
\]
\[
    R_2 + (-\frac{5}{3}R_3) = 
    \left[
    \begin{array}{ccc|ccc}
        0 & 1 & 0 & -\frac{21}{3} & -\frac{9}{3} & 5
    \end{array}
    \right]
\]
\[
    \left[
    \begin{array}{ccc|ccc}
        1 & 0 & -\frac{16}{3} & \frac{5}{3} & -\frac{2}{3} & 0 \\
        0 & 1 & 0 & -7 & -3 & 5 \\
        0 & 0 & 1 & 4 & 2 & -3
    \end{array}
    \right]
\]
\[
    R_1 + (\frac{16}{3}R_3) = 
    \left[
    \begin{array}{ccc|ccc}
        1 & 0 & 0 & \frac{69}{3} & \frac{30}{3} & \frac{-48}{3}
    \end{array}
    \right]
\]
\[
    \left[
    \begin{array}{ccc|ccc}
        1 & 0 & 0 & 23 & 10 & -16 \\
        0 & 1 & 0 & -7 & -3 & 5 \\
        0 & 0 & 1 & 4 & 2 & -3
    \end{array}
    \right]
\]
\[
    B^{-1} = \begin{bmatrix}
        23 & 10 & -16 \\
        -7 & -3 & 5 \\
        4 & 2 & -3
    \end{bmatrix}
\]

\subsection*{6e. C = \(
    \begin{bmatrix}
        1 & a & b + c \\
        1 & b & a + c \\
        1 & c & a + b
    \end{bmatrix}
    \)}
\[
    \left[
    \begin{array}{ccc|ccc}
        1 & a & b + c & 1 & 0 & 0 \\
        1 & b & a + c & 0 & 1 & 0\\
        1 & c & a + b & 0 & 0 & 1
    \end{array}
    \right]
\]
\[
    R_2 + (-R_1) = 
    \left[
    \begin{array}{ccc|ccc}
        0 & b - a & a - b & -1 & 1 & 0
    \end{array}
    \right]
\]
\[
    \left[
    \begin{array}{ccc|ccc}
        1 & a & b + c & 1 & 0 & 0 \\
        0 & b - a & a - b & -1 & 1 & 0\\
        1 & c & a + b & 0 & 0 & 1
    \end{array}
    \right]
\]
\[
    R_3 + (-R_1) = 
    \left[
    \begin{array}{ccc|ccc}
        0 & c - a & a - c & -1 & 0 & 1
    \end{array}
    \right]
\]
\[
    \left[
    \begin{array}{ccc|ccc}
        1 & a & b + c & 1 & 0 & 0 \\
        0 & b - a & a - b & -1 & 1 & 0\\
        0 & c - a & a - c & -1 & 0 & 1
    \end{array}
    \right]
\]
\[
    \frac{R_2}{b - a} = 
    \left[
    \begin{array}{ccc|ccc}
        0 & 1 & \frac{a - b}{b - a} & \frac{-1}{b - a} & \frac{1}{b - a} & 0
    \end{array}
    \right]
\]
\[
    \left[
    \begin{array}{ccc|ccc}
        1 & a & b + c & 1 & 0 & 0 \\
        0 & 1 & -1 & \frac{1}{a - b} & -\frac{1}{a - b} & 0\\
        0 & c - a & a - c & -1 & 0 & 1
    \end{array}
    \right]
\]
\[
    R_3 + -((c - a)R_2) = 
    \left[
    \begin{array}{ccc|ccc}
        0 & 0 & 0 & \frac{b - c}{a - b} & \frac{c - a}{a - b} & 1
    \end{array}
    \right]
\]
\[
    \left[
    \begin{array}{ccc|ccc}
        1 & a & b + c & 1 & 0 & 0 \\
        0 & 1 & -1 & \frac{1}{a - b} & -\frac{1}{a - b} & 0\\
        0 & 0 & 0 & \frac{b - c}{a - b} & \frac{c - a}{a - b} & 1
    \end{array}
    \right]
\]
Since \(R_3\) has no pivot variable, we cannot turn the variables above it (namely \(b + c\) and \(-1\)) to 0 to reduce it to RREF. Thus, the inverse does not exist.\pagebreak

\section*{7. Using inverses to solve systems}
\subsection*{7a. Inverse of \(
    \begin{pmatrix}
        2 & 5 & 8 & 5 \\
        1 & 2 & 3 & 1 \\
        2 & 4 & 7 & 2 \\
        1 & 3 & 5 & 3
    \end{pmatrix}
    \)}
\[
    \left[
    \begin{array}{cccc|cccc}
        2 & 5 & 8 & 5 & 1 & 0 & 0 & 0 \\
        1 & 2 & 3 & 1 & 0 & 1 & 0 & 0 \\
        2 & 4 & 7 & 2 & 0 & 0 & 1 & 0 \\
        1 & 3 & 5 & 3 & 0 & 0 & 0 & 1
    \end{array}
    \right]
\]
\[
    \frac{R_1}{2} = 
    \left[
    \begin{array}{cccc|cccc}
        1 & \frac{5}{2} & 4 & \frac{5}{2} & \frac{1}{2} & 0 & 0 & 0
    \end{array}
    \right]
\]
\[
    R_2 + (-R_1) = 
    \left[
    \begin{array}{cccc|cccc}
        0 & -\frac{1}{2} & -1 & -\frac{3}{2} & -\frac{1}{2} & 1 & 0 & 0
    \end{array}
    \right]
\]
\[
    R_3 + (-2R_1) = 
    \left[
    \begin{array}{cccc|cccc}
        0 & -1 & -1 & -3 & -1 & 0 & 1 & 0
    \end{array}
    \right]
\]
\[
    R_4 + (-R_1) = 
    \left[
    \begin{array}{cccc|cccc}
        0 & \frac{1}{2} & 1 & \frac{1}{2} & -\frac{1}{2} & 0 & 0 & 1
    \end{array}
    \right]
\]
\[
    \left[
    \begin{array}{cccc|cccc}
        1 & \frac{5}{2} & 4 & \frac{5}{2} & \frac{1}{2} & 0 & 0 & 0 \\
        0 & -\frac{1}{2} & -1 & -\frac{3}{2} & -\frac{1}{2} & 1 & 0 & 0 \\
        0 & -1 & -1 & -3 & -1 & 0 & 1 & 0 \\
        0 & \frac{1}{2} & 1 & \frac{1}{2} & -\frac{1}{2} & 0 & 0 & 1
    \end{array}
    \right]
\]
\[
    -2R_2 = 
    \left[
    \begin{array}{cccc|cccc}
        0 & 1 & 2 & 3 & 1 & -2 & 0 & 0
    \end{array}
    \right]
\]
\[
    R_3 + R_2 = 
    \left[
    \begin{array}{cccc|cccc}
        0 & 0 & 1 & 0 & 0 & -2 & 1 & 0
    \end{array}
    \right]
\]
\[
    R_4 + (-\frac{R_2}{2}) = 
    \left[
    \begin{array}{cccc|cccc}
        0 & 0 & 0 & -1 & -1 & 1 & 0 & 1
    \end{array}
    \right]
\]
\[
    \left[
    \begin{array}{cccc|cccc}
        1 & \frac{5}{2} & 4 & \frac{5}{2} & \frac{1}{2} & 0 & 0 & 0 \\
        0 & 1 & 2 & 3 & 1 & -2 & 0 & 0 \\
        0 & 0 & 1 & 0 & 0 & -2 & 1 & 0 \\
        0 & 0 & 0 & -1 & -1 & 1 & 0 & 1
    \end{array}
    \right]
\]
\[
    -R_4 = 
    \left[
    \begin{array}{cccc|cccc}
        0 & 0 & 0 & 1 & 1 & -1 & 0 & -1
    \end{array}
    \right]
\]
\[
    \left[
    \begin{array}{cccc|cccc}
        1 & \frac{5}{2} & 4 & \frac{5}{2} & \frac{1}{2} & 0 & 0 & 0 \\
        0 & 1 & 2 & 3 & 1 & -2 & 0 & 0 \\
        0 & 0 & 1 & 0 & 0 & -2 & 1 & 0 \\
        0 & 0 & 0 & 1 & 1 & -1 & 0 & -1
    \end{array}
    \right]
\]
\[
    R_2 + (-3R_4) = 
    \left[
    \begin{array}{cccc|cccc}
        0 & 1 & 2 & 0 & -2 & 1 & 0 & 3
    \end{array}
    \right]
\]
\[
    \left[
    \begin{array}{cccc|cccc}
        1 & \frac{5}{2} & 4 & \frac{5}{2} & \frac{1}{2} & 0 & 0 & 0 \\
        0 & 1 & 2 & 0 & -2 & 1 & 0 & 3 \\
        0 & 0 & 1 & 0 & 0 & -2 & 1 & 0 \\
        0 & 0 & 0 & 1 & 1 & -1 & 0 & -1
    \end{array}
    \right]
\]
\[
    R_1 + (-\frac{5}{2}R_4) = 
    \left[
    \begin{array}{cccc|cccc}
        1 & \frac{5}{2} & 4 & 0 & -2 & \frac{5}{2} & 0 & \frac{5}{2}
    \end{array}
    \right]
\]
\[
    \left[
    \begin{array}{cccc|cccc}
        1 & \frac{5}{2} & 4 & 0 & -2 & \frac{5}{2} & 0 & \frac{5}{2} \\
        0 & 1 & 2 & 0 & -2 & 1 & 0 & 3 \\
        0 & 0 & 1 & 0 & 0 & -2 & 1 & 0 \\
        0 & 0 & 0 & 1 & 1 & -1 & 0 & -1
    \end{array}
    \right]
\]
\[
    R_2 + (-2R_3) = 
    \left[
    \begin{array}{cccc|cccc}
        0 & 1 & 0 & 0 & -2 & 5 & -2 & 3
    \end{array}
    \right]
\]
\[
    \left[
    \begin{array}{cccc|cccc}
        1 & \frac{5}{2} & 4 & 0 & -2 & \frac{5}{2} & 0 & \frac{5}{2} \\
        0 & 1 & 0 & 0 & -2 & 5 & -2 & 3 \\
        0 & 0 & 1 & 0 & 0 & -2 & 1 & 0 \\
        0 & 0 & 0 & 1 & 1 & -1 & 0 & -1
    \end{array}
    \right]
\]
\[
    R_1 + (-4R_3) =
    \left[
    \begin{array}{cccc|cccc}
        1 & \frac{5}{2} & 0 & 0 & -2 & \frac{21}{2} & -4 & \frac{5}{2}
    \end{array}
    \right]
\]
\[
    \left[
    \begin{array}{cccc|cccc}
        1 & \frac{5}{2} & 0 & 0 & -2 & \frac{21}{2} & -4 & \frac{5}{2} \\
        0 & 1 & 0 & 0 & -2 & 5 & -2 & 3 \\
        0 & 0 & 1 & 0 & 0 & -2 & 1 & 0 \\
        0 & 0 & 0 & 1 & 1 & -1 & 0 & -1
    \end{array}
    \right]
\]
\[
    R_1 + (-\frac{5}{2}R_2) =
    \left[
    \begin{array}{cccc|cccc}
        1 & 0 & 0 & 0 & 3 & -2 & 1 & -5
    \end{array}
    \right]
\]
\[
    \left[
    \begin{array}{cccc|cccc}
        1 & 0 & 0 & 0 & 3 & -2 & 1 & -5 \\
        0 & 1 & 0 & 0 & -2 & 5 & -2 & 3 \\
        0 & 0 & 1 & 0 & 0 & -2 & 1 & 0 \\
        0 & 0 & 0 & 1 & 1 & -1 & 0 & -1
    \end{array}
    \right]
\]
The inverse is \(
    \begin{bmatrix}
        3 & -2 & 1 & -5 \\
        -2 & 5 & -2 & 3 \\
        0 & -2 & 1 & 0 \\
        1 & -1 & 0 & -1
    \end{bmatrix}\).

\subsection*{7b. Solve system of equations}
\(Ax = b\), where \(A = \begin{pmatrix}
                            2 & 5 & 8 & 5 \\
                            1 & 2 & 3 & 1 \\
                            2 & 4 & 7 & 2 \\
                            1 & 3 & 5 & 3
                        \end{pmatrix}\),
                    \(x = \begin{pmatrix}
                           x_1 \\
                           x_2 \\
                           x_3 \\
                           x_4 
                        \end{pmatrix}\),
                    \(b = \begin{pmatrix}
                        0 \\
                        1 \\
                        0 \\
                        1 
                     \end{pmatrix}\).\newline
We can solve for \(x\) by solving for \(A^{-1}b\).
\[
    A^{-1} =
    \begin{pmatrix}
        3 & -2 & 1 & -5 \\
        -2 & 5 & -2 & 3 \\
        0 & -2 & 1 & 0 \\
        1 & -1 & 0 & -1
    \end{pmatrix}
\]
\[
    A^{-1}b = 
    \begin{pmatrix}
        3(0) + -2(1) + 1(0) + -5(1) \\
        -2(0) + 5(1) + -2(0) + 3(1) \\
        0(0) + -2(1) + 1(0) + 0(1) \\
        1(0) + -1(1) + 0(0) + -1(1)
    \end{pmatrix}
\]
\[
    x = A^{-1}b = 
    \begin{pmatrix}
        -7 \\
        8 \\
        -2 \\
        -2
    \end{pmatrix}
\]


\end{document}
